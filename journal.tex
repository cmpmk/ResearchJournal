\documentclass[10pt]{article}
\usepackage[utf8]{inputenc}
\usepackage{tcolorbox}
\usepackage{amsmath}
\usepackage{titling}
\pdfpagewidth=8.5in \pdfpageheight=11in
\setlength{\droptitle}{-13em}
\title{Research Journal}
\author{Christopher Malek}
\date{}

\newcommand{\bt}{\begin{tcolorbox}}
\newcommand{\edtog}{$\epsilon_{d}/\epsilon_{g}$}
\begin{document}
\maketitle
\section{Preliminary}
\textbf{Streaming Instability Runs}\\
1) d2g\_test: set $\epsilon_{d}/\epsilon_{g} = 0.1$\\
2) res\_test: $N_{x} = 384$, $N_{z} = 32$, $L_{z} = 0.08772845953$\\
3) log\_test: grid\_func = 'log', temp\_powerlaw = 0.5\\
4) cs\_test: $N_{x} = 2048$, $N_{z} = 32$, $L_{z} = 0.0164142647777$, $cs_{0} = 0.1$\\

\noindent Then repeat 1-4 in 2D in the midplane: \edtog = 0.1

\section{05/01/18}
Positive pressure gradient causes particles to be pushed out. Three new tests to run:\\
1) d2g: switch gas backreaction off\\
2) cs: remove the inner particle boundary ($rp_{int}=0.6$ default)\\
3) midplane: set \edtog = 1.0\\

\noindent Flattening gradient kills Streaming Instability\\
axis=0 in Python selects the azimuthal direction\\

Plotting a 1D average against time: loop through VAR files, average both dust and gas\\
\* Fig. 7 from Lyra and Kuchner for reference.\\




\end{document}
